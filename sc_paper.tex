\documentclass{acm_proc_article-sp}

\usepackage{minted}
% \usemintedstyle{monokai}

\usepackage{quoting}
\quotingsetup{vskip=0pt}

\usepackage{hyperref}

\begin{document}

\title{Improving Software Installation Techniques with Smithy}

% \subtitle{[Extended Abstract]}
% \titlenote{A full version of this paper is available as
% \textit{Author's Guide to Preparing ACM SIG Proceedings Using
% \LaTeX$2_\epsilon$\ and BibTeX} at
% \texttt{www.acm.org/eaddress.htm}}}

\numberofauthors{2} %  in this sample file, there are a *total*
%
\author{
\alignauthor
Anthony DiGirolamo\\ %\titlenote{Dr.~Trovato insisted his name be first.}\\
       \affaddr{National Center for Computational Sciences}\\
       \affaddr{Oak Ridge National Laboratory}\\
       \affaddr{1 Bethel Valley Rd}\\
       \affaddr{Oak Ridge, TN USA}\\
       \email{anthonyd@ornl.gov}
\alignauthor
Robert D. French\\ %\titlenote{The secretary disavows any knowledge of this author's actions.}\\
       \affaddr{National Center for Computational Sciences}\\
       \affaddr{Oak Ridge National Laboratory}\\
       \affaddr{1 Bethel Valley Rd}\\
       \affaddr{Oak Ridge, TN USA}\\
       \email{frenchrd@ornl.gov}
}

\date{12 August 2015}

\maketitle

\begin{abstract}

Smithy is a software compilation and installation tool that borrows ideas from
SWTools\cite{swtools} and the homebrew\cite{homebrew} package management system
for Mac OS X.  Smithy is designed to manage many software builds within an HPC
Linux environment using modulefiles to load software into a user's shell.
SWTools has set very good conventions for software installations at the National
Center for Computational Sciences (NCCS).
Smithy's goal is to make following the SWtools conventions easier and less error
prone.  Smithy replaces SWTools and improves upon it by providing a simpler
command line interface, modulefile generation and management, and installations
via formulas written in Ruby.

\end{abstract}

% A category with the (minimum) three required fields
% \category{H.4}{Information Systems Applications}{Miscellaneous}
%A category including the fourth, optional field follows...
% \category{D.2.8}{Software Engineering}{Metrics}[complexity measures, performance measures]

% \terms{Theory}

% \keywords{ACM proceedings, \LaTeX, text tagging} % NOT required for Proceedings

\section{Introduction}

The sane management of third-party software installation requests is a vital
component of the user experience at any high-performance computing center. Users
expect installation requests to be handled promptly, and for required software
to be available on each computing resource they use. Unfortunately for the
software administrator, the disparity between computing resources can mean hours
of work deducing appropriate configuration options, testing for correct
behavior, or even crafting patches to support esoteric systems.

SWTools\cite{swtools} aided the software administrator by providing tools and
conventions to facilitate reproducible software builds. Information specific to
the configuration of a build could be kept in \textit{rebuild scripts} that
would house the information necessary to recompile and reinstall a package.
However, SWTools is very specific in its aims: a single rebuild script supports
one set of configuration options for a particular platform. This can be a
hindrance, for example, for packages like Python NumPy, which can target
multiple major versions of Python (2.X and 3.X) as well as multiple BLAS
implementations.  Supporting all combinations of Python and BLAS for a single
compute resource can require many rebuild scripts, each very similar.

Smithy improves over SWTools primarily in this regard. Drawing inspiration from
Homebrew\cite{homebrew}, a package manager for Mac OS X, Smithy encapsulates
everything necessary for reproducible builds in a single ruby program called a
``formula''. Formulas can be written in a generalized fashion to support
multiple software versions or build platforms; they can also be extended using
ruby's ``concerns'' mechanism to support separate configurations per version or
per platform when necessary.

\section{Installation and Configuration}

Installation is a simple download and extract anywhere affair. Smithy comes
bundled with its own x86, x86\_64, and Mac OS X builds of Ruby that will run on
most machines.  See the releases page for download tar files at \\
\href{http://github.com/AnthonyDiGirolamo/smithy/releases}{(github.com/AnthonyDiGirolamo/smithy/releases)}

Smithy is designed to manage many software installations across many hosts and
architectures. To support this it must be configured via a yaml formatted config
file. As an example:

\begin{quoting}
\begin{minted}{yaml}
---
software-root: /sw
hostname-architectures:
  rhea:        redhat6
  titan-login: xk7
  titan-ext:   xk7
file-group-name: ccsstaff
formula-directories:
- /sw/tools/smithy/formulas
download-cache: /sw/sources
\end{minted}
\end{quoting}

The above lines and many more configuration options are documented on the smithy
manpage at \\
\href{http://anthonydigirolamo.github.io/smithy/smithy.1.html}{(anthonydigirolamo.github.io/smithy/smithy.1.html)}.
Notable are the \texttt{software-root} and \texttt{hostname-architectures}
options. \texttt{software-root} is the root directory where all software will be
installed. \texttt{hostname-architectures} maps machine hostnames to specific
architecture directory names. For example when smithy is run on a machine with
the hostname \texttt{titan-ext1} it will place installations within the
\texttt{/sw/xk7} directory.

\section{Formulas}

Smithy formulas are Ruby programs that deduce platform-specific build settings.
They can manipulate the programming environment and load or unload modules as
appropriate. Formulas can express dependency requirements, and even generate
modulefiles for newly installed packages.

Formulas are installed with an SWTools target name. Targets are formatted as
\texttt{name/version/build\_name} and correspond to the directory structure used
to install the software on the filesystem.  For example a target can be
specified such as:

\begin{quoting}
\begin{verbatim}
python_numpy/1.9.2/python2.7_sles11.1_acml
\end{verbatim}
\end{quoting}

This tells smithy to load the \texttt{python\_numpy} formula, and build version
\texttt{1.9.2} targeting Python2.7 on SuSE Linux Enterprise Server version 11.1
using AMD Core Math Library. This is a suitable target for Titan, a Cray XK7
running SLES 11.1 with AMD Opteron CPUs. When installed Smithy will layout the
directory structure for the above target as:

\begin{quoting}
\begin{verbatim}
/sw/xk7
\___ python_numpy
   \___ 1.9.2
      \___ python2.7_sles11.1_acml
         |--- bin
         |--- lib
         \___ source
\end{verbatim}
\end{quoting}

The root location for all software installations is \texttt{/sw} and all
software installations for Titan are installed under the \texttt{xk7}
architecture subdirectory.

To install NumPy on Rhea, an NCCS analysis cluster running RedHat 6.5 on Intel
Xeon CPUs, Smithy can be invoked with the target:

\begin{quoting}
\begin{verbatim}
python_numpy/1.9.2/python2.7_rhel6.5_mkl
\end{verbatim}
\end{quoting}

Which corresponds to the following directory structure:

\begin{quoting}
\begin{verbatim}
/sw/rhea
\___ python_numpy
   \___ 1.9.2
      \___ python2.7_rhel6.5_mkl
         |--- bin
         |--- lib
         \___ source
\end{verbatim}
\end{quoting}

Smithy will invoke the same \texttt{python\_numpy} formula as before, but this
time targeting RHEL 6.5 and the Intel Math Kernel Library.

As the configuration options and strategy for building numpy are largely the
same between these two cases, the specifics needed to handle each can be
expressed in conditional blocks if they are simple, or in ``concerns'' if they
are more complex. This allows the software administrator to maintain portable
build instructions for dissimilar platforms in a single file.

When we say portable we mean that Smithy formulas should be share-able between
dissimilar architectures (Cray XK7, XE6, and XC-30) as well as related x86-Linux
clusters. That is, one formula to capture all installation caveats of a package
within the center.

See Figure 2 and Figure 3
% Figure~\ref{fig:PythonNumpyFormula1} and Figure~\ref{fig:PythonNumpyFormula2}
for the complete \texttt{python\_numpy} formula.

A further advantage of Smithy formulas over SWTools rebuild scripts, inspired by
Homebrew formulas, is that they can be shared through public version-control
hosting sites such as GitHub. HPC centers can choose to maintain their own
formulas, pull changes downstream from the official Smithy formulas repository,
or contribute additional improvements upstream.

\subsection{Formula Basics}

Formulas are written as Ruby classes and follow a simple structure. They
provide a domain specific language (DSL) with the support of a full programming
language. The structure of a formula is the same as a Ruby class. Here is a bare
minimum example:

\begin{quoting}
\begin{minted}{ruby}
class ExampleFormula < Formula
  homepage "http://example.com"
  url "http://example.com/example-1.2.3.tar.gz"
  md5 "44d667c142d7cda120332623eab69f40"

  def install
    system "./configure --prefix=#{prefix}"
    system "make install"
  end
end
\end{minted}
\end{quoting}

The above is a working example of a formula written in the Ruby programming
language. All examples shown are syntactically valid Ruby. No pseudo code is
used. The above formula can be installed via the command line by running:

Using the Ruby language to write formulas has been somewhat of a challenge at
our center. To help address that we have included links and general Ruby
programming information in the
\href{http://anthonydigirolamo.github.io/smithy/smithyformula.5.html#COMMON-OPERATIONS}{(Smithy formula writing documentation)}.

\begin{quoting}
\begin{minted}{sh}
$ smithy formula install example/1.2.3/gnu4.3.4
\end{minted}
\end{quoting}

Assuming that this is run on Titan, smithy will download the
\texttt{example-1.2.3.tar.gz} from the supplied url, check its md5 sum, extract
it to \texttt{/sw/xk7/example/1.2.3/gnu4.3.4/source}, switch to that working
directory, and run the commands specified via the \texttt{system} method.
Namely:

\begin{quoting}
\begin{minted}{sh}
$ ./configure \\
  --prefix=/sw/xk7/example/1.2.3/gnu4.3.4
$ make install
\end{minted}
\end{quoting}

The \texttt{homepage}, \texttt{url}, and \texttt{md5} DSL methods take a string
as their argument. The \texttt{def install...end} section is a Ruby method
definition that will be run after Smithy extracts the software and sets up the
shell environment. Inside that method any arbitrary Ruby code can be added to
perform the installation. The \texttt{system} method will run shell commands and
the \texttt{prefix} method is a string variable that Smithy will populate with
the full installation path based on the target name supplied to the
\texttt{smithy formula install} command.

\subsection{Formula Helpers}

Formulas have a wealth of additional helper features each of which is documented
on the formula writing manpage at \\
\href{http://anthonydigirolamo.github.io/smithy/smithyformula.5.html}{(anthonydigirolamo.github.io/smithy/smithyformula.5.html)}.
Here we will highlight some of the most useful ones.

\subsubsection{\texttt{concern for\_version()}}

Concerns are used allow formulas to support multiple versions. When used, their
contents will override the same methods defined at the root level of a formula
class. To specify a concern for version 1.2.3 of a package add the following to
a formula (replacing the comment with the methods you would like to override).

\begin{quoting}
\begin{minted}{ruby}
concern for_version("1.2.3") do
  included do
    # override methods here
  end
end
\end{minted}
\end{quoting}

Any formula DSL method can be overridden including \texttt{def install}.

Here is an example of a python formula that supports version ``2.7.9'' and
``3.4.3''. This example only overrides the download url and md5.

\begin{quoting}
\begin{minted}{ruby}
class PythonFormula < Formula
  homepage "www.python.org/"

  depends_on "sqlite"

  module_commands ["unload python"]

  concern for_version("2.7.9") do
    included do
      url "https://www.python.org/"
          "ftp/python/2.7.9/Python-2.7.9.tgz"
      md5 "5eebcaa0030dc4061156d3429657fb83"
    end
  end

  concern for_version("3.4.3") do
    included do
      url "https://www.python.org/"
          "ftp/python/3.4.3/Python-3.4.3.tgz"
      md5 "4281ff86778db65892c05151d5de738d"
    end
  end

  def install
    module_list
    ENV["CPPFLAGS"] =
      "-I#{sqlite.prefix}/include"
    ENV["LDFLAGS"] =
      "-L#{sqlite.prefix}/lib"
    system "./configure --prefix=#{prefix}",
           "--enable-shared"
    system "make install"
  end
end
\end{minted}
\end{quoting}

Using the above formula it is possible to install python with the following and
smithy will download the correct tarball for each version:

\begin{quoting}
\begin{minted}{sh}
$ smithy formula install \\
  python/2.7.9/sles11.3_gnu4.3.4
$ smithy formula install \\
  python/3.4.3/sles11.3_gnu4.3.4
\end{minted}
\end{quoting}

\subsubsection{\texttt{depends\_on}}

This method expects either a single string or an array of strings that define
dependencies for this formula. Here are some examples:

\begin{quoting}
\begin{minted}{ruby}
depends_on "curl"
depends_on [
  "cmake",
  "qt",
  "openssl",
  "sqlite" ]
\end{minted}
\end{quoting}

Using this method ensures that if a given dependency is not met smithy will
abort the installation. It also provides a way to query dependent packages
information within the install method later on. For example if you write
\texttt{depends\_on "curl"} in your formula you gain access to an object named curl
inside the install method. This allows you to do things like:

\begin{quoting}
\begin{minted}{ruby}
system "./configure --prefix=#{prefix}",
       "--with-curl=#{curl.prefix}"
\end{minted}
\end{quoting}

In the above example \texttt{\#\{curl.prefix\}} is an example of a ruby
interpolated string, everything between the \texttt{\#\{ \}} is ruby code.
\texttt{curl.prefix} will return a string with the location curl is installed
in.

The strings passed to \texttt{depends\_on} are just the locations of installed
software.  If you required a specific version of a dependency you could
specify the version or build numbers of existing installed software. For
example:

\begin{quoting}
\begin{minted}{ruby}
depends_on [
  "cmake/2.8.11.2/sles11.1_gnu4.3.4",
  "qt/4.8.5",
  "sqlite" ]
\end{minted}
\end{quoting}

Assuming your software root is \texttt{/sw/xk7} smithy would look for the above
software installs in \\
\texttt{/sw/xk7/cmake/2.8.11.2/sles11.1\_gnu4.3.4}, \\
~\texttt{/sw/xk7/qt/4.8.5/*}, and \texttt{/sw/xk7/sqlite/*/*}. The wildcard
\texttt{*} works similar to shell globbing. If you needed to install a python
module that depends on a specific version of another python module you might
use:

\begin{quoting}
\begin{minted}{ruby}
depends_on [ "python/3.3.0",
  "python_numpy/1.7.1/*python3.3.0*" ]
\end{minted}
\end{quoting}

This would require a given formula to have access to both
\texttt{/sw/xk7/python/3.3.0/*} and a python module with a build name that includes
\texttt{python3.3.0} located at \\
\texttt{/sw/xk7/python\_numpy/1.7.1/*python3.3.0*}

You may also need to specify dependencies conditionally upon the type of build
you are performing. If you add the type of build to the \texttt{build\_name}
when installing you can key off that to specify dependencies. Taking the python
example further, lets extend it to support multiple versions of python. You can
pass a ruby block to the \texttt{depends\_on} method to make it more dynamic.
The syntax for this is:

\begin{quoting}
\begin{minted}{ruby}
depends_on do
  # ...
end
\end{minted}
\end{quoting}

Any ruby code may go in here the last executed line of the block should be an
array of strings containing the dependencies. Lets use a ruby case statement
for this:

\begin{quoting}
\begin{minted}{ruby}
depends_on do
  case build_name
  when /python3.3/
    [ "python/3.3.0",
      "python_numpy/1.7.1/*python3.3.0*" ]
  when /python2.7/
    [ "python/2.7.3",
      "python_numpy/1.7.1/*python2.7.3*" ]
  end
end
\end{minted}
\end{quoting}

In this example case statement switches on the \texttt{build\_name}. The
\texttt{when /python3.3/} will be true if the \texttt{build\_name} contains the
\texttt{python3.3}. The
\texttt{/python3.3/} syntax is a regular expression.

This allows the formula to set it's dependencies based off the type of build
thats being performed. Lets say this formula is \texttt{python\_matplotlib}. You could
run either of these commands to install it and expect the dependencies to be set
correctly:

\begin{quoting}
\begin{minted}{sh}
$ smithy formula install \\
  python_matplotlib/1.2.3/python3.3.0
$ smithy formula install \\
  python_matplotlib/1.2.3/python2.7.3
\end{minted}
\end{quoting}

\subsubsection{\texttt{module\_commands}}

This method defines the module commands that must be run before \texttt{system}
calls within the \texttt{def install} part of the modulefile. Similar to
\texttt{depends\_on} it expects an array of strings with each string being a
module command. For example:

\begin{quoting}
\begin{minted}{ruby}
module_commands [ "load szip", "load hdf5" ]
\end{minted}
\end{quoting}

A more complicated example:

\begin{quoting}
\begin{minted}{ruby}
module_commands [
  "unload PE-gnu PE-pgi PE-intel PE-cray",
  "load PE-gnu",
  "load cmake/2.8.11.2",
  "load git",
  "swap gcc gcc/4.7.1",
  "swap ompi ompi/1.6.3"
]
\end{minted}
\end{quoting}

\texttt{module\_commands} also accepts ruby blocks the syntax for this is:

\begin{quoting}
\begin{minted}{ruby}
module_commands do
  # ...
end
\end{minted}
\end{quoting}

This can be used to dynamically set which modules to load based on the
\texttt{build\_name}. Here is an example that loads the correct python version:

\begin{quoting}
\begin{minted}{ruby}
module_commands do
  commands = [ "unload python" ]

  case build_name
  when /python3.3/
    commands << "load python/3.3.0"
  when /python2.7/
    commands << "load python/2.7.3"
  end

  commands << "load python_numpy"
  commands << "load szip"
  commands << "load hdf5/1.8.8"
  commands
end
\end{minted}
\end{quoting}

This block starts by creating a variable named \texttt{commands} as an array with a
single item \texttt{"unload python"}. Next a case statement is used to determine which
version of python we are compiling for. \mintinline{ruby}{commands << "load python/3.3.0"} will
append \texttt{"load python/3.3.0"} to the end of the array.  After that, it appends a
few more modules to load.  The last line of the block must be the array itself
so that when the block is evaluated by smithy, it recieves the expected value.

Assuming this is a formula for \texttt{python\_h5py} running \texttt{smithy
formula install python\_h5py/2.1.3/python3.3} results in an array containing:

\begin{quoting}
\begin{minted}{ruby}
[ "unload python",
  "load python/3.3.0",
  "load python\_numpy",
  "load szip",
  "load hdf5/1.8.8" ]
\end{minted}
\end{quoting}

\subsubsection{\texttt{modulefile}}

The modulefile command provides a way to define a complete modulefile within the
formula itself. It's recommended to have one modulefile per application and version combination.

Writing modulefiles is a topic in and of itself. For details on the modulefile
format see the
\href{http://modules.sourceforge.net/man/modulefile.html}{modulefile manpage}
Modulefiles are written in tcl and can take many forms.

Here is an example of a modulefile that points to a single build:

\begin{quoting}[leftmargin=0pt]
\begin{minted}{ruby}
modulefile <<-MODULEFILE
  #%Module
  proc ModulesHelp { } {
    puts stderr \
    "<%= @package.name %> <%= @package.version %>"
    puts stderr ""
  }
  module-whatis \
    "<%= @package.name %> <%= @package.version %>"

  set PREFIX <%= @package.prefix %>

  prepend-path PATH            $PREFIX/bin
  prepend-path LD_LIBRARY_PATH $PREFIX/lib
  prepend-path MANPATH         $PREFIX/share/man
MODULEFILE
\end{minted}
\end{quoting}

The \mintinline{ruby}{<<-MODULEFILE} syntax denotes the beginning of a
multi-line string. The string ends with \texttt{MODULEFILE}. You can substitute
any word for \texttt{MODULEFILE}.

The modulefile definition uses the \href{http://ruby-doc.org/stdlib-2.0/libdoc/erb/rdoc/ERB.html}{erb
format}. Anything
between the \mintinline{erb}{<%= ... %>} delimiters will be interpreted as ruby code. There are
a few helper methods that you can use inside these delimiters see the
\href{http://anthonydigirolamo.github.io/smithy/smithyformula.5.html}{``Modulefile
Helper Methods'' section of the smithyformula manpage}.

A more complicated modulefile may examine a user's programming environment to
determine which build to load. For instance if the user has gcc or a gnu
programming environment module loaded then your modulefile will want to load the
build compiled with the gcc compiler.  Here is an example designed to
dynamically set the build:

\begin{quoting}[leftmargin=0pt]
\begin{minted}{erb}
#%Module
proc ModulesHelp { } {
  puts stderr \
  "<%= @package.name %> <%= @package.version %>"
  puts stderr ""
}
# One line description
module-whatis \
  "<%= @package.name %> <%= @package.version %>"

<% if @builds.size > 1 %>
<%= module_build_list @package, @builds %>

set PREFIX \
  <%= @package.version_directory %>/$BUILD
<% else %>
set PREFIX <%= @package.prefix %>
<% end %>

# Helpful ENV Vars
setenv <%= @package.name.upcase %>_DIR \
  $PREFIX
setenv <%= @package.name.upcase %>_LIB \
  "-L$PREFIX/lib"
setenv <%= @package.name.upcase %>_INC \
  "-I$PREFIX/include"

# Common Paths
prepend-path PATH            $PREFIX/bin
prepend-path LD_LIBRARY_PATH $PREFIX/lib
prepend-path MANPATH         $PREFIX/share/man
prepend-path INFOPATH        $PREFIX/info
prepend-path PKG_CONFIG_PATH $PREFIX/lib/pkgconfig
\end{minted}
\end{quoting}

The main difference from the first example is the \\
\mintinline{erb}{<%= if @builds.size > 1 %>}
block. This basically checks to see if we have installed multiple builds or
not. If that condition is true everything up until the
\mintinline{erb}{<% else %>}
will be put
in the modulefile including the output from the \\
\texttt{module\_build\_list}
method. Otherwise, if we have only one build, \mintinline{erb}{set PREFIX <%= @package.prefix %>} will be put in the
  modulefile.

\subsubsection{\texttt{module\_build\_list}}

This is a helper method that will generate the TCL necessary to conditionally
load builds based on what compiler programming environment modules a user has
loaded. It takes \texttt{@package} and \texttt{@builds} as arguments.
For example, building the package ``bzip2'' for the GNU, PGI, and Intel compiler
environments with the following builds

\begin{quoting}[leftmargin=0pt]
\begin{minted}{sh}
$ smithy formula install bzip2/1.0.4/gnu4.3.4
$ smithy formula install bzip2/1.0.4/gnu4.7.2
$ smithy formula install bzip2/1.0.4/pgi13.4
$ smithy formula install bzip2/1.0.4/intel12
\end{minted}
\end{quoting}

will generate the following TCL conditional to load the correct build for the
currently loaded compiler environment:

\begin{quoting}[leftmargin=0pt]
\begin{minted}{tcl}
if [ is-loaded PrgEnv-gnu ] {
  if [ is-loaded gcc/4.3.4 ] {
    set BUILD gnu4.3.4
  } elseif [ is-loaded gcc/4.7.2 ] {
    set BUILD gnu4.7.2
  } else {
    set BUILD gnu4.7.2
  }
} elseif [ is-loaded PrgEnv-pgi ] {
  set BUILD pgi13.4
} elseif [ is-loaded PrgEnv-intel ] {
  set BUILD intel12
}
if {![info exists BUILD]} {
  puts stderr "[module-info name] is only
  available for the following environments:"
  puts stderr "gnu4.3.4"
  puts stderr "gnu4.7.2"
  puts stderr "intel12"
  puts stderr "pgi13.4"
  break
}
\end{minted}
\end{quoting}

\section{Similar Efforts}

Smithy is not the only tool aiming to facilitate portable software builds. Some
other tools include EasyBuild\cite{EasyBuild}, Spack\cite{spack}, and
HashDist\cite{hashdist}.

\subsection{Smithy's Goals}

Smithy was designed to support building software on Cray systems and related
clusters. One of the main challenges of supporting software on Cray systems is
dealing with complex Cray-specific build environments that can require
cross-compilation and restrict the usage of shared libraries. Another challenge
is the generation of environment modulefiles that are aware of compiler and MPI
environments. We have not found this combination of features in other tools.

In our experience, software packages which do not specifically mention Cray
support do not necessarily build and install correctly without
modification. This is where using a full programming language such as Ruby helps
in terms of patching, configuration, or writing out Cray-specific files in a
readable and organized way. The advantage of using Ruby over markup formats such
as YAML (as in HashDist) is the ability to make complex choices about how to
satisfy dependencies at build time. This is also a strength of Spack and
EasyBuild's use of Python.

HashDist and Spack do not support modulefile generation. EasyBuild does, but it
does not provide support for Cray-specific build requirements.

\subsection{EasyBuild}

EasyBuild supports a wide variety of scientific applications, over 500 packages
in total. The Smithy formulas repository \\
\href{http://github.com/olcf/smithy\_formulas}{(github.com/olcf/smithy\_formulas)}
contains formulas for over 150 packages written by multiple staff members
including PETSc, OpenMPI, NetCDF, Magma, a large collection of host-native
Python modules including NumPy and SciPy, as well as several domain-specific
packages such as LAMMPS, NAMD, and Gromacs.

While far short of the number boasted by EasyBuild, Smithy formulas excel over
EasyBuild's ``easyblocks'' in two regards: they emphasize support for esoteric
platforms such as Crays, and since formulas are contained in a single file, they
are easier to write and maintain.

\subsection{Spack}

Spack provides similar capabilities to Smithy, but is implemented in Python. It
also boasts some features not available in Smithy such as auto dependency
resolution and pre-bundled formulas that are generally installable on most Linux
systems. In general, Smithy formulas are geared towards handling the caveats
that arise on niche systems such as large Crays.

\subsection{HashDist}

HashDist contains around 450 supported packages at the time of this writing
including some that support Cray machines.  HashDist makes writing packages (or
``specs'') easy by using YAML markup to define how to install a piece of
software using a defined structure. This approach may seem attractive, but given
the detailed knowledge of HashDist's capabilities required to write effective
packages, the flexibility provided by using a full-featured programming language
allows Smithy to handle edge cases more gracefully.

\section{Conclusions}

Smithy formulas provide a self-contained encapsulation of all logic necessary to
build multiple versions of a scientific software package for various resources,
operating systems, and programming environments. A software administrator
employing smithy can support build requests for many disparate compute resources
within the same center. The concise nature of the Ruby programming language
allows formula authors to spare themselves the tedium of repetitive software
build tasks. The formula helper methods provided by the Smithy formula DSL
facilitate dealing with the caveats of deploying software to these environments.

Formulas are not a new idea in package management, they are similar to Homebrew
formulas with additional functionality to support HPC specific requirements such
as multiple compilers and environment modulefiles. They are analogous to Spack's
packages or EasyBuild's EasyBlocks, with the advantage that all logic needed for
an installation is contained in a single file. When managing multiple systems of
varying architectures, with multiple compiler environments per system,
minimizing the number of files to manage is a boon to the software maintainer's
sanity.

Although Smithy does not support as many packages as other tools at the time of
writing, it is the authors' belief that Smithy's ease of adoption and rich
feature set will allow HPC centers to collaborate, grow the number of packages,
and maintain the existing ones.

\section{Acknowledgments}

The authors would like to thank the US Department of Energy's Office of
Science's Advanced Scientific Computing Research program.

\section{Further Reading}

The Smithy Homepage
\href{http://anthonydigirolamo.github.io/smithy/}{(anthonydigirolamo.github.io/smithy)}
has information on installing, configuring and running smithy on your system. In
addition there are two in depth manpages that cover using smithy on the command
line
\href{http://anthonydigirolamo.github.io/smithy/smithy.1.html}{(anthonydigirolamo.github.io/smithy/smithy.1.html)}
and formula writing
\href{http://anthonydigirolamo.github.io/smithy/smithyformula.5.html}{(anthonydigirolamo.github.io/smithy/smithyformula.5.html)}.

% \begin{description}
%   \item[Homepage] \href{http://anthonydigirolamo.github.io/smithy/}{http://anthonydigirolamo.github.io/smithy/}
%   \item[Usage Manpage] \href{http://anthonydigirolamo.github.io/smithy/smithy.1.html}{http://anthonydigirolamo.github.io/smithy/smithy.1.html}
%   \item[Formula Manpage] \href{http://anthonydigirolamo.github.io/smithy/smithyformula.5.html}{http://anthonydigirolamo.github.io/smithy/smithyformula.5.html}
%   % \item[Slides] \href{http://anthonydigirolamo.github.io/smithy/slides.html}{http://anthonydigirolamo.github.io/smithy/slides.html}
% \end{description}

\bibliographystyle{abbrv}
\bibliography{sc_paper}

% \begin{thebibliography}{1}

% \bibitem{swtools}
% N. Jones, M. R. Fahey, "Design, Implementation, and Experiences of Third-Party
% Software Administration at the ORNL NCCS," Proceedings of the 50th Cray User
% Group (CUG08), Helsinki, May 2008.

% \bibitem{homebrew}
% Homebrew. brew.sh. Retrieved December, 2013 from http://brew.sh/

% \end{thebibliography}

\appendix

\begin{figure*} % 2 column span
\begin{minted}[linenos]{ruby}
class GitFormula < Formula
  homepage "https://git-core.googlecode.com/"
  url      "https://github.com/git/git/archive/v2.3.2.tar.gz"
  sha256   "7d8e15a2f41b8d6c391e527f461d61027cf3391c9ccc89b8c1a1a0785f18a0fb"

  depends_on ["curl/7.39.0", "zlib"]

  def install
    module_list
    system "make configure"
    system "./configure --prefix=#{prefix}",
           "--with-curl=#{curl.prefix}",
           "--with-zlib=#{zlib.prefix}"
    system "make install"

    system "mkdir -p #{prefix}/share/man"
    system "curl -O",
      "https://www.kernel.org/pub/software/scm/git/git-manpages-2.3.2.tar.gz"
    system "cd #{prefix}/share/man && ",
      "tar xf #{prefix}/source/git-manpages-2.3.2.tar.gz"
  end

  modulefile do
    <<-MODULEFILE
      #%Module
      proc ModulesHelp { } {
         puts stderr "<%= @package.name %> <%= @package.version %>"
         puts stderr ""
      }
      # One line description
      module-whatis "<%= @package.name %> <%= @package.version %>"

      set PREFIX <%= @package.prefix_directory %>

      prepend-path PATH      $PREFIX/bin
      prepend-path PERL5LIB  $PREFIX/lib64/perl5/site_perl
      prepend-path MANPATH   $PREFIX/share/man
      setenv       GITDIR    $PREFIX
    MODULEFILE
  end
end
\end{minted}
\caption{Example Git Formula}
\label{fig:GitFormula}
\end{figure*}

\begin{figure*} % 2 column span
\begin{minted}[linenos]{ruby}
class PythonNumpyFormula < Formula
  homepage "http://www.numpy.org/"
  additional_software_roots [ config_value("lustre-software-root").fetch(hostname) ]

  supported_build_names /python.*_gnu.*/

  concern for_version("1.8.0") do
    included do
      url "http://downloads.sourceforge.net/project/numpy/NumPy/1.8.0/numpy-1.8.0.tar.gz"
    end
  end

  concern for_version("1.9.2") do
    included do
      url "http://downloads.sourceforge.net/project/numpy/NumPy/1.9.2/numpy-1.9.2.tar.gz"
    end
  end

  depends_on do
    [ python_module_from_build_name, "cblas/20110120/*acml*" ]
  end

  module_commands do
    pe = "PE-"
    pe = "PrgEnv-" if cray_system?

    commands = [ "unload #{pe}gnu #{pe}pgi #{pe}cray #{pe}intel" ]
    commands << "load #{pe}gnu"
    commands << "swap gcc gcc/#{$1}" if build_name =~ /gnu([\d\.]+)/

    commands << "load acml"
    commands << "unload python"
    commands << "load #{python_module_from_build_name}"

    commands
  end

  def install
    module_list

    acml_prefix = module_environment_variable("acml", "ACML_BASE_DIR")
    acml_prefix += "/gfortran64"

    FileUtils.mkdir_p "#{prefix}/lib"
    FileUtils.cp "#{cblas.prefix}/lib/libcblas.a", "#{prefix}/lib", verbose: true
    FileUtils.cp "#{acml_prefix}/lib/libacml.a",   "#{prefix}/lib", verbose: true
    FileUtils.cp "#{acml_prefix}/lib/libacml.so",  "#{prefix}/lib", verbose: true

    ENV['CC']  = 'gcc'
    ENV['CXX'] = 'g++'
    ENV['OPT'] = '-O3 -funroll-all-loops'
\end{minted}
\label{fig:PythonNumpyFormula1}
\caption{Example Python NumPy Formula (1 of 2)}
\end{figure*}

\begin{figure*}
\begin{minted}[linenos,firstnumber=52]{ruby}
    File.open("site.cfg", "w+") do |f|
      f.write <<-EOF.strip_heredoc
        [blas]
        blas_libs = cblas, acml
        library_dirs = #{prefix}/lib
        include_dirs = #{cblas.prefix}/include

        [lapack]
        language = f77
        lapack_libs = acml
        library_dirs = #{acml_prefix}/lib
        include_dirs = #{acml_prefix}/include

        [fftw]
        libraries = fftw3
        library_dirs = /opt/fftw/3.3.0.1/x86_64/lib
        include_dirs = /opt/fftw/3.3.0.1/x86_64/include
      EOF
    end

    system "cat site.cfg"

    system_python "setup.py build"
    system_python "setup.py install --prefix=#{prefix} --compile"
  end

  modulefile do
    <<-MODULEFILE
      #%Module
      proc ModulesHelp { } {
         puts stderr "<%= @package.name %> <%= @package.version %>"
         puts stderr ""
      }
      # One line description
      module-whatis "<%= @package.name %> <%= @package.version %>"

      prereq PrgEnv-gnu PE-gnu
      prereq python
      conflict python_numpy

      <%= python_module_build_list @package, @builds %>
      set PREFIX <%= @package.version_directory %>/$BUILD

      set LUSTREPREFIX \
        #{additional_software_roots.first}/#{arch}/<%= @package.name %>/<%= @package.version %>/$BUILD

      prepend-path LD_LIBRARY_PATH $LUSTREPREFIX/lib
      prepend-path LD_LIBRARY_PATH $LUSTREPREFIX/lib64
      prepend-path PYTHONPATH      $LUSTREPREFIX/lib/$LIBDIR/site-packages
      prepend-path PYTHONPATH      $LUSTREPREFIX/lib64/$LIBDIR/site-packages

      prepend-path PATH            $PREFIX/bin
      prepend-path LD_LIBRARY_PATH $PREFIX/lib
      prepend-path LD_LIBRARY_PATH $PREFIX/lib64
      prepend-path LD_LIBRARY_PATH /opt/gcc/4.8.2/snos/lib64
      prepend-path LD_LIBRARY_PATH /ccs/compilers/gcc/rhel6-x86_64/4.8.2/lib
      prepend-path LD_LIBRARY_PATH /ccs/compilers/gcc/rhel6-x86_64/4.8.2/lib64
      prepend-path MANPATH         $PREFIX/share/man

      prepend-path PYTHONPATH      $PREFIX/lib/$LIBDIR/site-packages
      prepend-path PYTHONPATH      $PREFIX/lib64/$LIBDIR/site-packages
    MODULEFILE
  end
end
\end{minted}
\label{fig:PythonNumpyFormula2}
\caption{Example Python NumPy Formula (2 of 2)}
\end{figure*}

\end{document}
